%%%%%%%%%%%%%%%%%%%%%%%%%%%%%%%%%%%%%%%%%
% Beamer Presentation
% LaTeX Template
% Version 1.0 (10/11/12)
%
% This template has been downloaded from:
% http://www.LaTeXTemplates.com
%
% License:
% CC BY-NC-SA 3.0 (http://creativecommons.org/licenses/by-nc-sa/3.0/)
%
%%%%%%%%%%%%%%%%%%%%%%%%%%%%%%%%%%%%%%%%%

%----------------------------------------------------------------------------------------
%	PACKAGES AND THEMES
%----------------------------------------------------------------------------------------

\documentclass{beamer}

\mode<presentation> {

% The Beamer class comes with a number of default slide themes
% which change the colors and layouts of slides. Below this is a list
% of all the themes, uncomment each in turn to see what they look like.

%\usetheme{default}
%\usetheme{AnnArbor}
%\usetheme{Antibes}
%\usetheme{Bergen}
%\usetheme{Berkeley}
%\usetheme{Berlin}
%\usetheme{Boadilla}
%\usetheme{CambridgeUS}
%\usetheme{Copenhagen}
%\usetheme{Darmstadt}
%\usetheme{Dresden}
%\usetheme{Frankfurt}
%\usetheme{Goettingen}
%\usetheme{Hannover}
%\usetheme{Ilmenau}
%\usetheme{JuanLesPins}
%\usetheme{Luebeck}
\usetheme{Madrid}
%\usetheme{Malmoe}
%\usetheme{Marburg}
%\usetheme{Montpellier}
%\usetheme{PaloAlto}
%\usetheme{Pittsburgh}
%\usetheme{Rochester}
%\usetheme{Singapore}
%\usetheme{Szeged}
%\usetheme{Warsaw}

% As well as themes, the Beamer class has a number of color themes
% for any slide theme. Uncomment each of these in turn to see how it
% changes the colors of your current slide theme.

%\usecolortheme{albatross}
%\usecolortheme{beaver}
%\usecolortheme{beetle}
%\usecolortheme{crane}
%\usecolortheme{dolphin}
%\usecolortheme{dove}
%\usecolortheme{fly}
%\usecolortheme{lily}
%\usecolortheme{orchid}
%\usecolortheme{rose}
%\usecolortheme{seagull}
%\usecolortheme{seahorse}
%\usecolortheme{whale}
%\usecolortheme{wolverine}

%\setbeamertemplate{footline} % To remove the footer line in all slides uncomment this line
%\setbeamertemplate{footline}[page number] % To replace the footer line in all slides with a simple slide count uncomment this line

%\setbeamertemplate{navigation symbols}{} % To remove the navigation symbols from the bottom of all slides uncomment this line
}

\usepackage{graphicx} % Allows including images
\usepackage{booktabs} % Allows the use of \toprule, \midrule and \bottomrule in tables
\usepackage{hyperref}

\hypersetup{colorlinks=false}

%----------------------------------------------------------------------------------------
%	TITLE PAGE
%----------------------------------------------------------------------------------------

\title[Free Software]{A Brief Overview of Free Software} % The short title appears at the bottom of every slide, the full title is only on the title page

\author{George Wilson} % Your name
\institute[Griffith University] % Your institution as it will appear on the bottom of every slide, may be shorthand to save space
{
Griffith University \\ % Your institution for the title page
\medskip
\href{http://www.github.com/gwils}{www.github.com/gwils} \\
\href{http://www.twitter.com/GeorgeTalksCode}{@GeorgeTalksCode} % Your email address
}
\date{\today} % Date, can be changed to a custom date

\begin{document}

\begin{frame}
\titlepage % Print the title page as the first slide
\end{frame}

\begin{frame}
\frametitle{Overview} % Table of contents slide, comment this block out to remove it
\tableofcontents % Throughout your presentation, if you choose to use \section{} and \subsection{} commands, these will automatically be printed on this slide as an overview of your presentation
\end{frame}

%----------------------------------------------------------------------------------------
%	PRESENTATION SLIDES
%----------------------------------------------------------------------------------------

%------------------------------------------------
\section{What Free Software is} % Sections can be created in order to organize your presentation into discrete blocks, all sections and subsections are automatically printed in the table of contents as an overview of the talk
%------------------------------------------------

\subsection{Vocabulary} % A subsection can be created just before a set of slides with a common theme to further break down your presentation into chunks

\begin{frame}
\frametitle{``Free''?}
\begin{itemize}
\item English is a terrible language!
\item Free as in {\bf Freedom}, not as in {\bf free beer}.
\item I will use ``gratis'' to refer to things without monetary cost.
\end{itemize}

\end{frame}

%------------------------------------------------

\subsection{Definition}

\begin{frame}
\frametitle{What is Free Software?}

Free software is software that respects users' freedom and community.
Software is free if the user of the software has all four essential freedoms:

\begin{itemize}
\item {\bf Freedom 0} is the freedom to run the program as you wish, 
  for any purpose.
\item {\bf Freedom 1} is the study and modify the program. 
  Access to the source code is a precondition to this.
\item {\bf Freedom 2} is the freedom to redistribute exact copies of the 
  software, so that you can help your neighbour.
\item {\bf Freedom 3} is the freedom to distribute copies of your 
  modified versions to others. Access to the source code is a precondition 
  to this.
\end{itemize}
\end{frame}

%------------------------------------------------

\section{What Proprietary Software is}

\begin{frame}
\frametitle{What Proprietary Software is}

We call software which is not free, ``proprietary software''.
Proprietary software does not respect the freedom or community of the users.

The software has an ``owner'' who has power over all the users.
There can be no collaborative community; users are divided and 
powerless. Any changes they want in the software must be included by
the ``owner''.


\end{frame}

%------------------------------------------------

\begin{frame}
\frametitle{Example Practices}

The following are examples of freedom-denying practices which make software 
proprietary:

\begin{itemize}
\item Restricting what the user can use the software for;
\item Restricting the user from access to the source code;
\item Giving the user access to the source code, but restricting them 
  from modifying it or using parts of it in their program; and
\item Disallowing the copying and distribution of the software.
\end{itemize}

\end{frame}

%------------------------------------------------

\section{Free Software and Open Source Software}

\begin{frame}
\frametitle{Free Software and Open Source Software}

There is a significant overlap between software which is ``free'' 
and software which is ``open source'', but there is also a significant 
difference between these two movements.

\begin{itemize}
\item Free software is about the users' {\bf freedom}.
\item Open source is about the developers' {\bf practical benefit}.
\item Proponents of open source software may encourage similar practices 
  to proponents of free software without making the freedoms clear.
\end{itemize}

\end{frame}

%------------------------------------------------

\section{Copyleft and Software Licenses}

\begin{frame}
\frametitle{Copyleft}

\ldots

\end{frame}


%------------------------------------------------

\begin{frame}
\frametitle{Software Licenses}

\ldots

\end{frame}

%------------------------------------------------

\begin{frame}
\Huge{\centerline{Questions?}}
\end{frame}

%------------------------------------------------

\begin{frame}
\Huge{\centerline{The End}}
\end{frame}

%----------------------------------------------------------------------------------------

\end{document} 
